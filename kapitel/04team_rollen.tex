\chapter{Aufgaben - Kompetenzen - Verantwortlichkeiten (AKV)}\label{ch:AKV}
Um komplikationsfrei das Projekt durchführen zu können, ist es ein essentieller Schritt innerhalb
des Teams jedem Gruppenmitglied eine oder mehrere Rollen zuzuweisen.
Damit sämtliche Erfahrungen und Fähigkeiten aller Gruppenmitglieder optimal genutzt werden können, 
ist es wichtig die Rollen so zu verteilen, dass jeder seine Stärken einbringen kann. 
Dies trägt dazu bei die Zeit in den Arbeitszeitblöcken effizient zu nutzen, indem keine der aktuellen Aufgaben
mehrfach bearbeitet werden. 
Demnach ergibt sich folgende Rollenverteilung, die innerhalb der folgenden AKV-Matrizen näher erläutert wird.

\vspace{1cm}
\begin{table}[H]
    \begin{center}
        \label{tab:projektmanager}
        \begin{tabular}{c|p{9cm}}
            Aufgaben & Kontrolle aller tabellarischen Dokumente sowie des \\
            & Zeitmanagements inklusive Einhaltung der Fristen, \\
            & Dokumentation der Aufwandsschätzung und des\\
            & Risikomanagements, Qualitätskontrolle der Software \\
            & und Management des Entwicklungsablaufs \\
            \hline
            Kompetenzen & Erstellung der Tabellen (Excelkenntnisse), \\
            & Erinnerung anderer Gruppenmitglieder an Fristen, \\
            & Qualitätskriterien beachten bei der Entwicklung, \\
            & Aufwände präsent halten \\
            \hline
            Verantwortlichkeiten & Korrekten Verlauf des Projekts sicherstellen, \\
            & Einhaltung von Fristen, regelmäßiges Prüfen aller  \\
            & Dokumentationen und Managementtabellen \\
        \end{tabular}
        \caption{AKV-Matrix des Projektmanagers}
    \end{center}
\end{table}
\begin{table}[H]
    \begin{center}
        \label{tab:devopsengineer}
        \begin{tabular}{c|p{9cm}}
            Aufgaben & Implementierung von Github Actions über YAML- \\
            & Dateien mit entsprechender Erstellung von \\
            & Automatisierungsmechanismen des Kompilierungs- \\ 
            & vorgangs von Projekt und Paper, Testung der\\
            & implementierten Actions mit anschließender \\
            & Überprüfung der erstellten Binärdateien und der \\
            & korrekten Erstellung von Github Releases \\
            \hline
            Kompetenzen & Erstellen von Github Actions und Workflows,  \\
            & Verstehen des Syntax von Workflow YAML-Dateien, \\
            & Debugging von fehlgeschlagenen Testbuilds \\
            \hline
            Verantwortlichkeiten & Infrastruktur zur Kompilierung des Projekts und \\
            & des Papers zur automatisierten Bereitstellung von \\
            & PDF- und Binärdateien \\
        \end{tabular}
        \caption{AKV-Matrix des Dev-Ops-Engineers}
    \end{center}
\end{table}
\begin{table}[H]
    \begin{center}
        \label{tab:softwareentwickler}
        \begin{tabular}{c|p{9cm}}
            Aufgaben & Entwicklung der definierten Features in GODOT, \\
            & Schreiben von Skripten in C\# und SQL mit \\
            & anschließender Pull-Request von der eigenen \\
            & zur main-Branch, Benutzung von erstellten Assets\\
            & für Engine-spezifische Objekte, Einbindung von \\
            & Shadern und Animationen, Reviewen und Testen des Code \\
            \hline
            Kompetenzen & Erstellung und Bearbeitung sowie sicherer Umgang \\
            & mit Branches und Pull-Requests auf Github, selbst- \\
            & ständiges Management des Kanbansystems auf \\
            & Github, Spieleentwicklung mit der GODOT- \\
            & Engine, Überprüfung von Code in Pull-Request \\
            \hline
            Verantwortlichkeiten & Funktionierende Implementierung der Features \\
            & mit Bug-Fixing, Einhaltung der Frist für die \\
            & Entwicklung des Spiels, Gewährleistung von \\
            & Codequalität \\
        \end{tabular}
        \caption{AKV-Matrix des Softwareentwicklers}
    \end{center}
\end{table}
\begin{table}[H]
    \begin{center}
        \label{tab:paperautor}
        \begin{tabular}{c|p{9cm}}
            Aufgaben & Verfassen von Texten und Tabellen für das Paper, \\
            & Korrekturarbeit bei den von anderen Paperautoren \\
            & geschriebenen Kapiteln oder Texten \\
            \hline
            Kompetenzen & Umgang mit LaTeX und Tabellen im LaTeX-Format, \\
            & Verwaltung des Paperversionierung auf Github über \\
            & Branches und Pull-Requests \\
            \hline
            Verantwortlichkeiten & Fristgemäße Fertigstellung des Papers in korrekter \\
            & deutscher Rechtschreibung und Grammatik, \\
            & Formatierung des Papers \\
        \end{tabular}
        \caption{AKV-Matrix des Paperautors}
    \end{center}
\end{table}
\begin{table}[H]
    \begin{center}
        \label{tab:designer}
        \begin{tabular}{c|p{9cm}}
            Aufgaben & Entwerfen und Zeichnen der Assets für Spieltexturen \\
            & wie Pflanzen, Töpfe, Hintergründe, Button-Icons und \\
            & und Objektrahmen, Sammeln von Pflanzenarten für die Datenbank \\
            \hline
            Kompetenzen & Erstellung von Assets für Texturen mit Pixelart- \\
            & Tool, Hochladen der Assets auf Github in den \\
            & Projektordner, Recherche zu Pflanzen, \\
            & Höchstmaß an Kreativität \\
            \hline
            Verantwortlichkeiten & Bereitstellung von Assets für Texturen zur nahtlosen \\
            & Entwicklung des Spiels und Pflanzenfakten zur \\
            & Speicherung in der Datenbank \\
        \end{tabular}
        \caption{AKV-Matrix des Designers}
    \end{center}
\end{table}

\vspace{5cm}

Mit den zuvor beschrieben Rollen werden alle Aufgaben im Projekt Grow Green abgedeckt und
den jeweiligen Gruppenmitgliedern zugeordnet. 
Die Zuordnung wird nach Vorwissen, Erfahrungen, Interessen und Fähigkeiten getroffen. 
Die folgenden zwei Rollen werden besonders achtsam zugeteilt, da sie den Kern der Entwicklung des Spiels darstellen.
So wurden Gruppenmitgliedern die Rolle des Softwareentwicklers zugeordnet, falls diese sowohl Interesse
an der Programmierung und Erstellung des Spiels haben, als auch Erfahrungen mit der 
objektorientierten Programmierung.
Die Rolle des Designers wird nur denjenigen zugeordnet, die das höchste Maß an Kreativität der Gruppe besitzen
zum Erzielen der bestmöglichen Texturen des Spiels.
In folgender Matrix wird die Rollenverteilung des Grow Green-Teams dargestellt.


\begin{table}[H]
    \begin{center}
        \label{tab:rollenverteilung}
        \begin{tabular}{c|p{2cm}|p{2cm}|p{2cm}|p{2cm}|p{2cm}}
            Rollen 
            & Alexander 
            & Maja 
            & Theo 
            & Josh 
            & Domenik \\
            & Betke
            & Günther
            & Leuthardt
            & Tischer
            & Wilhelm \\
            \hline
            Projektmanager &  & x &  &  &  \\
            Dev-Ops-Engineer & x &  & x &  &  \\
            Softwareentwickler & x &  & x & x & x \\
            Paperautor &  & x & x &  &  \\
            Designer &  & x &  &  &  \\
        \end{tabular}
        \caption{Matrix zur Rollenverteilung}
    \end{center}
\end{table}