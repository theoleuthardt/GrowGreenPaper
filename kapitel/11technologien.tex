\chapter{Technologien und Produkte}\label{ch:technologien}
Zur Entwicklung des Projekts Grow Green werden Technologien verwendet, die der erleichterten Arbeitsweise dienen
und die Produktivität der Gruppe steigern sollen. 
Im Folgenden werden die verwendeten Technologien näher erläutert. \\
\newline
% Godot & C#
Zur Entwicklung des Spiels ist die initiale Auswahl der Engine sehr bedeutend. 
Wir haben uns für die Engine Godot in der 4.\ Version entschieden, da es sich um ein Open-Source-Projekt der Godot 
Foundation handelt und damit die Anforderung der Kostenfreiheit erfüllt.
Durch die Unterstützung der Programmiersprache C\#, neben der Godot-spezifischen Skriptsprache, stand Godot im Vergleich
zu anderen Engines nicht schlechter da und biete die Möglichkeit bisher erworbenes Wissen aus dem letzten Semester in 
objektorientierter Programmierung anwenden und erweitern zu können.
Zum leichteren Debugging sollte unsere verwendete Engine eine gute Dokumentation haben, was Godot für seine aktuelle 
Version in großen Umfang pflegt und ebenfalls ein entscheidendes Kriterium darstellte.
Zudem bietet Godot eine hohe Kompatibilität an bezüglich des späteren Exports des Spiels, womit das Spiel im 
Deployment für alle großen Betriebssysteme exportiert werden kann. \\
\newline
% SQL
Für das Backend kommt SQL zum Einsatz in Form einer Datenbank.
Dies stellt für uns die einfachste Möglichkeit dar, Daten verständlich zu speichern, ohne ein großes Backendframework
lernen zu müssen und damit einen größeren Zeitaufwand zu haben. 
Ebenfalls kann wie bei C\# aus den vergangenen Semestern gelerntes Wissen angewendet werden und somit zusätzliche 
Mehrarbeit vermieden werden.
% IDE
% Git & Github
% Github Actions
% Pixilart (Assets pixeln)
% Latex