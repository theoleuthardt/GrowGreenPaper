\chapter{Technologien und Produkte}\label{ch:technologien}
Zur Entwicklung des Spiels wurden Technologien ausgewählt, die der erleichterten Arbeitsweise dienen
und die Produktivität der Gruppe steigern sollen. \\
\newline
% Godot & C#
Die initiale Auswahl der Engine sehr bedeutend für das ganze Projekt. 
Wir haben uns für die Engine Godot in der 4. Version entschieden, da es sich um ein Open-Source-Projekt der Godot 
Foundation handelt und damit keien Kosten anfallen.
Neben der Godot-spezifischen Skriptsprache GDScript wird auch die Programmiersprache C\# unterstützt, was die Engine vergleichbar mit anderen Produkten macht.
Das bietet dem Team die Möglichkeit bisher erworbenes Wissen aus dem 
letzten Semester anzuwenden und zu erweitern.
Zur Erleichterung des Debugging sollte die verwendete Engine über eine ausführliche Dokumentation verfügen. 
Die aktuelle Godot Version umfast eine großumfängliche und stets gepflegte Dokumentation.
Zudem bietet Godot eine hohe Kompatibilität bezüglich des späteren Exports des Spiels an, womit das 
Deployment für alle gängigen Betriebssysteme ermöglicht wird. \\
\newline
% IDE
Als Entwicklungsumgebung wird, zusammen mit dem Godot-Editor, der Code-Editor Visual Studio Code verwendet. 
Durch Debugging-Tools, besseres Syntax-Highlighting und syntaktische Code-Vervollständigung eignet sich dieser besser als der Godot-Editor.
Visual Studio Code ist außerdem kostenfrei und bietet eine Vielzahl an Erweiterungen, die das Entwickeln des Spiels
vereinfachen.
Alternativ wurde auch die IDE Rider vom Unternehmen Jetbrains genutzt.
Die HWR stellt eine Jetbrains Lizenz zur Verfügung, weshalb erneut keine Mehrkosten entstehen.
Rider unterstützt alle Funktionalitäten von Visual Studio Code und beinhaltet weitere Tools. So entsteht durch die Nutzung der Umgebungen weder ein Vor- noch ein Nachteil.\\
\newline
% SQL
Für das Backend wurde eine SQL-Datenbank verwendet.
So können Daten verständlich gespeichert werden, ohne ein großes Backendframework
zu erschaffen und den Zeitaufwand exponentiell zu vergrößern. 
Ebenfalls kann, wie bei C\#, Wissen aus den vergangenen Semestern angewendet und somit zusätzliche 
Mehrarbeit vermieden werden. \\
\newpage
% Git & Github (Versionierung)
Zur parallelen Bearbeitung der Software, wird Git als Version-Control-System verwendet.
Git ist ein Open-Source-Programm, welches die Versionsverwaltung von Dateien und Projekten ermöglicht.
Durch die Verwendung von Git können Änderungen des entsprechenden Bearbeiters nachvollzogen und bei Bedarf rückgängig 
gemacht werden.
So arbeitet jeder Entwickler auf seinem eigenen Branch und kann Features darin abgetrennt vom Main-Branch 
implementieren.
Wird ein Feature fertiggestellt, fügt der Entwickler seinen Branch zum Main-Branch hinzu.
Github erweitert Git als Cloud-basierte Plattform die Versionierung um weitere Projektmanagement-Tools, zum
Beispiel durch ein Kanban-Board für TO-DOs und Issues, sowie Projektdokumentation und CI/CD-Features über Actions. 
Github ist ebenso kostenlos, deshalb wurde schnell entschlossen die Versionierung und das Deployment mit Github
zu realisieren.\\
\newline
% Github Actions (Deployment)
Für das Deployment des Spiels wird Github Actions verwendet, da es sich um eine integrierte CI/CD-Lösung handelt, die
kostenfrei von Github angeboten wird.
Die Idee den Deploymentprozess zu automatisieren entstand, um nicht jedes Mal manuell auf einem der 
Entwicklersysteme das Spiel exportieren und veröffentlichen zu müssen.
Über selbst erstellte Workflows können Bearbeiter des Github-Projekts Prozesse automatisieren.
Dabei wird bei jeder Änderung des Main-Branchs ein neuer Build des Spiels und ein Release erstellt, der auf der Github-Seite des Projekts veröffentlicht.
Dazu gibt es eine Liste der 
Änderungen und eine automatisch generierte Versionsnummer.\\
\newline
% Pixilart (Assets pixeln)
Damit GrowGreen einen einheitlichen Stil besitzt, wurde zu Beginn des Projekts entschieden eigene
Assets und Texturen für das Spiel zu erstellen.
Für die Erstellung der Assets wird die Webseite Pixilart verwendet, da sie eine einfache und kostenfreie Möglichkeit
bietet, Pixel-Art-Texturen zu designen.
Die Webseite enthält eine Vielzahl an Tools, wie zum Beispiel eine Farbpalette, verschiedene
Pinsel und Ebenen damit Designer effizient arbeiten können.
Die Assets werden direkt auf der Webseite erstellt und exportiert, was die Erstellung von eigenen
Spieltexturen vereinfacht und einen effizienten Erstellungsprozess ermöglicht.\\
\newline
% Latex (Paper)
Das zugehörige Paper wird in \LaTeX{}geschrieben, da es sich um ein Open-Source-Textsatzsystem handelt, welches die
Erstellung von Dokumenten erleichtert.
Innerhalb der Gruppe wurden bisherige Praxistransferberichte in \LaTeX{} geschrieben und so 
konnten die Gruppenmitglieder erneut auf bereits existierendes Wissen zurückgreifen und Arbeitsaufwand reduzieren.
In einem separaten Github Repository wird das Paper versioniert und ebenfalls über Github Actions automatisiert 
kompiliert und veröffentlicht.