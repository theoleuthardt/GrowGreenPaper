\addchap{Anhang}\label{ch:anhang}

\begin{table}[H]\label{tab:quality}
    \footnotesize
    \begin{tabular}{|c|l|l|c|l|l|c|c|l|l|l|}
        \hline
        Nr. & P-Nr. & P-Titel & QP & Q-Beschreibung & Toleranz & VW & DF & PT & IT & S \\[0.5ex]
        \hline\hline
        01 & 1. & GrowGreen & AT & Spielen möglich & & 1-5 & 1, 2 & 33.33. & 33.33. & A \\
        \hline
        02 & 1.3 & Oberfläche & FP & Test auf korrekte &  & 1-5 &  &  &  & A \\
           &       &            &    & Frontend- &   &   & & & &\\
           &       &            &    & Darstellung und &   &   & & & &\\
           &       &            &    & Erreichbarkeit der &   &   & & & &\\
           &       &            &    & Szenen &   &   & & & &\\
        \hline
        03 & 1.3.1 & Münzenanzahl & FP & Test der korrekten &  & 1 &  &  &  & A \\
        &       &            &    & Anzeige der &   &   & & & &\\
        &       &            &    & Münzen im Besitz &   &   & & & &\\
        \hline
        04 & 1.3.2 & Verlassen Button & FP & Test auf korrekten &  & 5 &  &  &  & A \\
        &       &            &    & Szenenwechsel &   &   & & & &\\
        \hline
        05 & 1.3.3 & Home Interface & FP & Die Hauptszene & Die Pflanzenanzahl & 1, 5 &  &  &  & A \\
        &       &            &    & des Spiels soll auf &  ist begrenzt. Mehr  &   & & & &\\
        &       &            &    & alle Möglichkeiten & Plätze können  &   & & & &\\
        &       &            &    & des Spiels verwei- & erspielt werden.  &   & & & &\\
        &       &            &    & sen und ist der &   &   & & & &\\
        &       &            &    & Hauptstellplatz &   &   & & & &\\
        &       &            &    & für Pflanzen. &   &   & & & &\\
        \hline
        06 & 1.3.3.1 & Gekaufte Pflanzen & FP &  &  & 1 &  &  &  & A \\
        \hline
        07 & 1.3.3.2 & Buttons & FP & Test auf korrekten &  & 1, 5 &  &  &  & A \\
        &       &            &    & Szenenwechsel &   &   & & & &\\
        \hline
        08 & 1.3.3.2.1 & Greenhouse & FP & Test auf korrekten &  & 5 &  &  &  & A \\
        &       &            &    & Szenenwechsel &   &   & & & &\\
        \hline
        09 & 1.3.3.2.2 & Shop & FP & Test auf korrekten &  & 5 &  &  &  & A \\
        &       &            &    & Szenenwechsel &   &   & & & &\\
        \hline
        10 & 1.3.3.2.3 & Minispiele & FP & Test auf korrekten &  & 2, 4 &  &  &  & A \\
        &       &            &    & Szenenwechsel &   &   & & & &\\
        \hline
        11 & 1.3.3.2.4 & Töpfe & FP &  &  & 5 &  &  &  & A \\
        \hline
        12 & 1.3.4 & Shop Interface & FP &  &  & 5 &  &  &  & A \\
        \hline
        13 & 1.3.4.1 & Kaufen & FP &  &  & 5 &  &  &  & A \\
        \hline
        14 & 1.3.4.2 & Schwierigkeit & KL &  &  & 1, 2 &  &  &  & A \\
        \hline
        15 & 1.3.5 & Minispiel Interface & FP &  &  & 2, 4 &  &  &  & A \\
        \hline
        16 & 1.3.5.1 & Auswahl & FP &  &  & 5 &  &  &  & A \\
        \hline
        17 & 1.3.5.2 & Namen & KL &  &  & 1, 2 &  &  &  & A \\
        \hline
        18 & 1.3.5.3 & Logos & FP &  &  & 2 &  &  &  & A \\
        \hline
        19 & 1.3.5.4 & Pack Interface & FP &  &  & 4 &  &  &  & A \\
        \hline
        20 & 1.3.5.4.1 & Pflanze/Topf & FP &  &  & 4 &  &  &  & A \\
        \hline
        21 & 1.3.5.4.2 & Kaufen & FP &  &  & 4 &  &  &  & A \\
        \hline
        22 & 1.3.5.5 & Memory Interface & FP &  &  & 2, 4 &  &  &  & A \\
        \hline
        23 & 1.3.5.5.1 & Schwierigkeit & FP &  &  & 2 &  &  &  & A \\
        \hline
    \end{tabular}
    \caption{Qualitätsregister \\ Legende: \\
    \textbf{QP} = Qualitätsprüfmethode, \\
    AT - Abschlusstest, FP - Funktionsprüfung, KL - Korrekturlesen \\
    \textbf{VW} - Verantwortlichkeit, \textbf{DF} - Durchführung\\
    1 - Alexander, 2 - Maja, 3 - Theo, 4 - Josh, 5- Domenik \\
    \textbf{PT} - Plan-Termin, \textbf{IT} - Ist-Termin\\
    \textbf{S} - Status, A - abgeschlossen, NA - nicht abgeschlossen \\}
\end{table}

\begin{table}[H]\label{tab:risk}
    \footnotesize
    \begin{tabular}{|c|l|l|l|l|l|l|l|}
        \hline
        Nr. & Risikobeschreibung & P-E & Auswirkung & Art & Typ & Behandlung & Beschreibung der Behandlung \\[0.5ex]
        \hline \hline
        R1 & \textbf{Minispiele} werden als & 60\% & gering & zeitlich & Chance & Ergreifen & Arbeitsteilung, regelmäßige\\
        & seperate &  &  &  &  && Checks, gegenseitige  \\
        & Komponente des &  &  & & &  & Hilfestellung innerhalb \\
        & Spiels erstellt und &  &&  &  &  & der Gruppe \\
        & sind über das Haus &  &  &  & & &  \\
        & erreichbar &  &  &  &  &&  \\
        \hline
        R2 & \textbf{Texturen von} & 75\% & schwer & zeitlich & Bedro- & Akzeptieren & Arbeitsteilung, regelmäßige \\
        & \textbf{Pflanzen} werden &  &&  & hung &  & Checks, gegenseitige  \\
        & nicht zur Fertig- &  &  &  & & & Hilfestellung innerhalb \\
        & stellung des Grund- &  &  &  &&  & der Gruppe \\
        & spiels fertig designt &  &  &  & & &  \\
        \hline
        R3 & \textbf{Texturen von Töpfen} & 10\% & mittel & zeitlich & Bedro- & Akzeptieren & Arbeitsteilung, regelmäßige \\
        & werden nicht zur &  &  &  &hung&  & Checks, gegenseitige \\
        & Fertigstellung des &  &  & & &  & Hilfestellung innerhalb \\
        & Grundspiels fertig &  &  &  &&  & der Gruppe \\
        & designt &  &  &  &  &&  \\
        \hline
        R4 & \textbf{Erweiterung des} & 30\% & gering & techno- & Bedro- & Eventualplan & Aufgabenpriorität vermindern \\
        & \textbf{Hauses um ein} &  &  & logisch & hung&  & und gegebenenfalls nicht \\
        & \textbf{Gewächshaus} wird &  &  & & &  & implementieren \\
        & nicht mit &  &  &  &  & & \\
        & entsprechender &  &  &  &&  &  \\
        & Freischaltungslogik &  &  &&  &  &  \\
        & fertig gestellt &  &  &  & & &  \\
        \hline
        R5 & Funktion fehlt, dass & 70\% & mittel & techno- & Bedro- & Vermeiden & Feature wird im Spiel nicht \\
        & der Spieler &  &  & logisch &hung  &  &umgesetzt \\
        & \textbf{Ernährungspunkte} &&  &  &  &  &  \\
        & \textbf{durch den Sauerstoff} &&  &  &  &  &  \\
        & \textbf{seiner Pflanzen} &  &  &&  &  &  \\
        & sammelt &  &  &  &  &  &\\
        \hline
        R6 & \textbf{Open World} fehlt, in & 80\% & gering & techno- & Chance & Vermeiden &  Feature wird im Spiel nicht \\
        & der der Spieler &  &  & logisch& &  & umgesetzt \\
        & zwischen seinem &  &  &  &&  &  \\
        & Haus, dem Shop und &  &  &&  &  &  \\
        & der Minispielhalle &  &  & & &  &  \\
        & wechselt &  &  &  &  & & \\
        \hline
        R7 & \textbf{Ausfall eines} & 30\% & gering & sozial & Bedro- & Eventualplan & Aufgabenverteilung innerhalb \\
        & \textbf{Gruppenmitglieds} & & &  & hung &  & der Gruppe umstrukturieren \\
        & durch Krankheit &  &  &  & & & über das Github Kanban \\
        \hline
        R8 & \textbf{Ausfall von} & 10\% & gering & sozial & Bedro- & Eventualplan & Die Bereitstellung des Spiels \\
        & \textbf{Deployment-} &  &  && hung &  & zur Abgabe auf Moodle kann \\
        & \textbf{Infrastruktur} auf & & &  &  &  & auch lokal auf einem der \\
        & Github &  &  &  & & & Entwickler-PCs ausgeführt \\
        &  &  &  &  & & & werden \\
        \hline
    \end{tabular}
    \caption{Risikoregister \\ Legende: \\ \textbf{P-E} = Eintrittswahrscheinlichkeit}
\end{table}