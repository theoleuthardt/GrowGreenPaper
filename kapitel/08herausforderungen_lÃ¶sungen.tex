\chapter{Herausforderungen und Lösungen}\label{ch:herausforderungen}
Die erste Herausforderungen zu Beginn des Projekts bestand für das Team darin, das sich auf ein Thema und eine Stilart
geeinigt wurde, da verschiedene Erfahrungs- und Wissensstände der Teammitglieder zu Projektideen mit unterschiedlichen 
Schwierigkeitsstufen geführt hätten. 
So wurde beschlossen, ein 2D-Pixelart-Spiel zu entwickeln, anstatt von der 3D-Spieleentwicklung Gebrauch zu machen, weil 
nur ein Teammitglied damit Erfahrung besaß und die Entwicklung eines solchen Spiels mit zugehörigen Texturen und Modellen
eine zu hohen zeitlichen Aufwand dargestellt hätte. 
Entscheidend war zudem die umfangreiche Recherche zur Auswahl der Engine zur Entwicklung des Spiels. 
Kriterien dafür waren eine hoffentlich steile Lernkurve mit dem Engine-Editor, die Möglichkeit zur Programmierung mit 
weitläufig verbreiteten Programmiersprache und eine gute Dokumentation zur effizienteren Einarbeitung und Beseitigung
von Fehlern. 