\chapter{Herausforderungen und Lösungen}\label{ch:herausforderungen}
% Projektthema und Technologieauswahl
Die erste Herausforderungen zu Beginn des Projekts bestand für das Team darin, das sich auf ein Thema und eine Stilart
geeinigt wurde, da verschiedene Erfahrungs- und Wissensstände der Teammitglieder zu Projektideen mit unterschiedlichen 
Schwierigkeitsstufen geführt hätten. 
So wurde beschlossen, ein 2D-Pixelart-Spiel zu entwickeln, anstatt von der 3D-Spieleentwicklung Gebrauch zu machen, weil 
nur Josh damit Erfahrung besaß und die Entwicklung eines solchen Spiels mit zugehörigen Texturen und Modellen
eine zu hohen zeitlichen Aufwand dargestellt hätte. 
Entscheidend war zudem die umfangreiche Recherche zur Auswahl der Engine zur Entwicklung des Spiels. 
Kriterien dafür waren eine hoffentlich steile Lernkurve mit dem Engine-Editor, die Möglichkeit zur Programmierung mit 
weitläufig verbreiteten Programmiersprache und eine gute Dokumentation zur effizienteren Einarbeitung und Beseitigung
von Fehlern. 
Mit der Nutzung der Open-Source-Engine GODOT und der Programmiersprache C\# wurden alle der geforderten 
Kriterien erfüllt, die für die Entwicklung des Projekts erforderlich sind. \\
\newline
% Frontend
Beim Frontend des Spiels lief der Entwicklungsprozess relativ reibungslos ab unter anderem durch den nutzerfreundlichen 
GODOT-Editor.
Viel mehr kamen Schwierigkeiten auf durch das Verwenden und Manipulieren von Objekten aus Godot in C\#.
In jeder Szene des Spiels existiert eine Hierarchie der sich darin befindenden Objekte, die die Struktur der Szene
darstellen. 
Bei dem Versuch zum Beispiel das Spiel zu schließen, wenn im Titelmenü der Ausschaltbutton gedrückt wird, muss im 
zugewiesenen Skript das Spiel geschlossen werden. 
Nur stellte sich die Frage, wie wir in C\# auf die Godot-Objekte zugreifen können. 
Durch die integrierte Godot-System-Klasse kann auf die zuvor genannte Hierarchie zugegriffen werden über relative Pfade
und so als verwendbares Objekt manipuliert werden kann oder wie im genannten Beispiel das Spiel durch einen 
systemspezifischen Befehl geschlossen werden kann.\\
\newline
% Backend
Damit während des Spielens die Daten des Nutzers und alle Pflanzendaten nicht beim Schließen des Spiels 
verloren gehen, entstand das Problem der Datenspeicherung mithilfe eines Backends. 
Möglich wäre die Nutzung von verschiedenste Backed-Frameworks gewesen wie ASP .NET oder gängige 
Datenbank-Wrapper-Systeme mit Javascript oder Java wie NodeJS oder Spring-Boot.
Da wir aber kein großes Spiel mit Skalierbarkeit entwickeln werden, ist es am effizientesten eine Datenbank in SQL zu 
definieren und das Wrapper-System in C\# direkt zu implementieren. 
Zusätzlicher Vorteil dabei war die Erfahrung von Alex und Maja bezüglich Datenbanken, was zu einer zügigen Entwicklung
des Backends führte.\\
\newline
\newpage
% Charaterauswahl
In Verbindung mit dem Backend erwies es sich als etwas schwieriger, die Charakterauswahl aus dem Titelmenü in anderen
Spielszenen zu speichern und demnach die Identifizierung der Auswahl.
Als Lösung wurde jedem Charakter eine ID zugeordnet, die in der Datenbank gespeichert wird und bei benötigtem Aufruf
in der jeweiligen Klasse durch den Datenbank-Wrapper geladen wird.
Die Assets der Charaktere enthalten ihre ID im Dateinamen, wodurch das Laden über GODOT deutlich vereinfacht ist. \\
\newline
% Organisation
Im generellen Sinne kamen die meisten Probleme in der Organisation der Entwicklung und Zusammenarbeit auf. 
Unter anderem regelmäßig Termine zu finden für Besprechungen außerhalb der Vorlesungen und einen einheitlichen
Kommunikationskanal zu haben, damit Vorstellungen zur Entwicklung des Spiels jedes Gruppenmitglieds global in der Gruppe
bekannt sind.
Zudem fehlte es anfangs, dass jeder über seine Aufgaben und Pflichten Bescheid weiß, was zu ineffizienter Arbeitsweise 
führte.
Um einen geordneten Entwicklungsprozess zu erhalten, entschieden wir uns Aufgaben und die Zuweisung von Bearbeitern 
in der Gruppe über ein Github-Kanban-System zu lösen.
Darin kann jedes Gruppenmitglied Issues erstellen, also sprich Aufgaben, die Features des Spiels oder ähnliche 
projektbezogene To-dos darstellen und Gruppenmitgliedern zugewiesen werden können.
Über die Kommentarfunktion innerhalb der Issues kann auch ein Austausch zu Problemen stattfinden,
wodurch die Kommunikation nicht nur auf die Besprechungen begrenzt ist.
Stellt ein Gruppenmitglied seine Aufgabe fertig, trägt jener seine benötigte Zeit in einer Excel-Tabelle ein, damit
die Arbeitszeit am Projekt verfolgt werden kann.