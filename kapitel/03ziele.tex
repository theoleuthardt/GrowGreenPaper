\chapter{Ziele}\label{ch:ziele}
Um den Erfolg eindeutig messen zu können, wurden präzise Anforderungen an das Projekt gestellt.
Nach mehreren Bearbeitungen und Veränderungen ergaben sich folgende Ziele.
\vspace{0.5cm}
\begin{table}[H]\label{tab:psteckbrief}
    \centering
    \begin{tabular}{|c|l|l|l|l|}
        \hline
        \textbf{Nr.} & \textbf{Klassi-} & \textbf{Beschreibung} & \textbf{Messkriterium} & \textbf{Priorität}\\
        & \textbf{fizierung} &&&\\[0.5ex]
        \hline \hline
        1 & Hauptziel & Fertigstellung des & Alle Kapitel wurden fertig gestellt mit & Hauptziel\\
        && Papers & mit zusätzlicher Unterzeichnung der ehren- &\\
        &&& wörtlichen Erklärung aller Gruppenmitglieder.&\\
        \hline
        2 & Leistung & Titelmenü (Startmenü & Zugehörige Menüpunkte zum Starten, Laden & Muss\\
        && des Spiels) & eines Spiels, zur Charakterauswahl und & \\
        &&& und zu den Einstellungen wurden erstellt,&\\
        &&& dessen Logik implementiert und getestet. & \\
        \hline
        3 & Leistung & Hauptszene & Hintergrund und Logik zu Spielerbewegung & Muss\\
        &&& sowie Pflanzenplazierlogik wurde implementiert. & \\
        \hline
        4 & Leistung &Shopszene& Logik zum Kauf von Pflanzensamen und & Muss\\
        &&& Shopinterface wurde implementiert.&\\
        \hline
        5 & Leistung & Implementierung der & Bewässerung und Wachstum wurde nach der & Muss\\
        && Pflanzenlogik & Implementierung getestet. &\\
        \hline
        6 & Leistung & Backend in Form einer & Spieler und Pflanzendaten werden korrekt & Muss\\
        && Datenbank zur Spei- & gespeichert und ausgelesen, Debuggingtools &\\
        && cherung aller Daten & dienen zum Testen der Werte. &\\
        \hline
        7 & Leistung & Texturen für Pflanzen, & Speicherung aller benötigten Assets auf & Muss\\
        && Töpfe und weitere & Github mit entsprechender Verfügbarkeit in &\\
        && Assets des Spiels & der Spielengine &\\
        \hline
        8 & Leistung & Automatisiertes Deploy- & Testweises Kompilieren des aktuellen & Muss\\
        && ment mit Github Actions & Stands von Paper und Spiel auf Github&\\
        \hline
        9 & Termin & Endpräsentation & Vorstellung des Projekts & Muss\\
        \hline
        10 & Termin & Abschluss des Projektes & Pünktliche Abgabe des Source-Codes & Soll\\
        &&& und der Endpräsentation auf Moodle am &\\
        &&& 29.10.2024 und des Papers am 05.11.2024&\\
        \hline
        11 & Leistung & Logoerstellung & Nutzung eines fertigen Logos in allen & Soll\\
        &&& Facetten des Spiel mit der Zufriedenheit &\\
        &&& aller Gruppenmitglieder &\\
        \hline
        12 & Leistung & Sehr gute Bewertung & Gewünschte Bewertung vom & Soll\\
        &&  des Projektes & Stakeholder wird erreicht.&\\
        \hline
        13 & Sozial & Lerneffekt bei Nutzern& Feedback von Nutzern & Soll\\
        \hline
        14 & Sozial & Entwicklung von  & Feedback von Nutzern des Spiels & Soll\\
        && Gewohnheiten&&\\
        \hline
        15 & Termin & Fertigstellung der & Das Spiel wird am 29.10.2024 um 23:59 Uhr & Kann\\
        && Spieleentwicklung & fertig gestellt in seiner Entwicklung. &\\
        \hline
    \end{tabular}\label{tab:projektziele}
    \caption{Projektziele}
\end{table}