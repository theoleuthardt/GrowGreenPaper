\chapter{Qualitätskriterien}\label{ch:quality}
Um die Qualität des Spiels zu bewerten und zu verbessern wurde ein Qualitätsregister erstellt. 
Mithilfe der Auflistung klarer Anforderungen kann zu Beginn des Projekts genau beschrieben
werden, welche Kriterien das Spiel zur Fertigstellung erfüllen muss und wie sie geprüft werden.\\
\newline
Jede der Anforderungen besitzt eine Qualitätsnummer, Prüfmethode, Beschreibung, Toleranz mit den jeweiligen 
Verantwortlichen, die Termine zu planmäßiger und realem Abschlussdatum und dem Status.
Zur Gliederung aller Anforderungen hat jede eine Qualitätsnummer, die in aufsteigender Reihenfolge vergeben wurde.
Anhand der zugehörigen Prüfmethode wird festgelegt, wie die Anforderung nach erfolgreicher Realisierung geprüft wird, 
ob durch eine visuelle, praktische oder andere Prüfmethode.
Des Weiteren gibt die Beschreibung den Inhalt der Anforderung kurz und prägnant wieder, um sie zu verstehen
und eindeutig einzuordnen. 
Die Toleranz gibt an, wie viel Spielraum bei der Prüfung gegeben ist, um es als erfüllt gelten 
zu lassen.
Die Verantwortlichen sind die Personen, die für die Erfüllung der Anforderung verantwortlich gemacht werden können,
im Falle einer gescheiterten Realisierung oder Prüfung und dienen demnach dem Controlling.
Dagegen spiegelt die durchführenden Personen nur diejenigen wieder, die die Anforderung realisiert und geprüft haben.
Als Letztes werden noch Daten zu Planung und tatsächlichem Abschluss der Realisierung aller Anforderungen festgehalten,
um die anfängliche Vorstellung gegen die Realität der Umsetzung gegenüberzustellen.
Der Status hängt von der Erfüllung der Prüfmethode ab und bestimmt ob die Anforderung abgeschlossen wurde oder noch in Bearbeitung ist.\\
\newline
In ~\autoref{tab:quality} %hier ref bitte
ist das Qualitätsregister zu finden, welches alle Qualitätskriterien des Spiels
enthält, sowie die dazugehörigen Informationen.