\chapter{Anleitung zur Verwendung}\label{ch:anleitung}
Das Spiel ist über das GitHub Repository oder die Datei in der Moodle Abgabe erhältlich. 
Unter dem Tab „Releases“ lässt sich im GitHub immer die neueste Version des Spieles für alle Betriebssysteme als 
ZIP-Datei finden. 
Diese muss heruntergeladen und an einen beliebigen, gut erreichbaren Ort entpackt werden. 
Danach kann das Spiel durch Ausführen der Exe Datei in dem entstandenen Ordner gestartet werden.\\
\newline
Zu Beginn des Spiels kann ein Charakter gewählt werden. 
Diese Wahl beeinflusst das Aussehen des Haupthauses. 
Wird ein neues Spiel erstellt, werden eventuell existierende, alte Speicherstände überschrieben und ein neuer angelegt.\\
\newline
Über den Shop können nun die ersten Pflanzen gekauft werden. 
Nach dem Kauf einer Pflanze wird diese automatisch auf einem freien Platz im Haus platziert. 
Standardmäßig haben alle Pflanzen einen brauen Topf, die Farbe kann jedoch über das Töpfe-Menü mit dem Rucksack Icon 
geändert werden. 
Dazu muss einfach ein Topf ausgewählt und dann durch Klicken auf eine Pflanze angewendet werden. 
Jede Topffarbe kann beliebig oft verwendet und die Farbe eines Topfes beliebig oft verändert werden. 
Platzierte Pflanzen können durch längeres gedrückt halten verschoben bzw.\ durch Verschieben in den Mülleimer, 
gelöscht werden.\\
\newline
Das Minigamesmenü bietet drei Auswahlmöglichkeiten. 
Aus den Plant-Packs lassen sich neue Pflanzen potenziell billiger erhalten als aus dem Shop. 
Topf-Packs schalten dagegen neue, einzigartige Spezialtöpfe frei, die auf vorhandene Pflanzen angewendet werden können. 
Durch Spielen des Memory-Minispiels, können außerdem weitere Münzen verdient werden, die dann z.B.\ im Shop benutzt 
werden können.\\
\newline
In der Haupthausszene befindet sich außerdem noch eine Tür in der Mitte. 
Diese kann für 100 Münzen geöffnet werden und schaltet den Zugang zum Greenhouse frei. 
Dieses funktioniert wie das Haupthaus und bietet dem Spieler einfach mehr Platz, um Pflanzen unterzubringen.\\
