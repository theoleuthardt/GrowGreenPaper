\chapter{Anleitung zur Verwendung}\label{ch:anleitung}
GrowGreen als ausführbare Datei ist über das GitHub Repository oder die Datei in der Moodle Abgabe erhältlich. 
Unter dem Tab „Releases“ wird in GitHub immer die aktuellste Version des Spiels veröffentlicht. Verfügbar für alle Betriebssysteme wird ist dort als 
ZIP-Datei zu finden. 
Diese muss heruntergeladen und an einem beliebigen, gut erreichbaren Speicherort entpackt werden. 
Das Spiel wird gestartetdurch Ausführen der .exe Datei im eben extrahierten Ordner.\\
\newline
Zu Beginn des Spiels kann ein Charakter gewählt werden. 
Diese Wahl beeinflusst die Einrichtung der Spielumgebung. 
Wird ein neues Spiel erstellt, werden bereits existierende, alte Speicherstände überschrieben und ein neuer angelegt.\\
\newline
Über den Shop können nun die ersten Pflanzen gekauft werden. 
Nach dem Kauf einer Pflanze wird diese automatisch auf einem freien Platz im Haus platziert. 
Standardmäßig haben alle Pflanzen einen brauen Terrakotta-Topf. Die Farbe kann über das Töpfe-Menü mit dem Rucksack-Icon 
geändert werden. 
Dazu muss einfach ein Topf ausgewählt und dann, durch Klicken, auf eine Pflanze angewendet werden. 
Jede Topffarbe kann beliebig oft verwendet und die Farbe eines Topfes beliebig oft verändert werden. 
Platzierte Pflanzen können durch längeres Klicken verschoben und durch Verschieben in den Mülleimer, 
gelöscht werden.\\
\newline
Das Minigame-Menü bietet drei Auswahlmöglichkeiten. 
Aus den Plant-Packs lassen sich neue Pflanzen zu unschlagbaren Preisen gewinnen. 
Pot-Packs schalten dagegen neue, einzigartige Spezialtöpfe frei, die auf vorhandene Pflanzen angewendet werden können. 
Durch Spielen des Memory-Minispiels, können außerdem weitere Münzen verdient werden, die dann beispielsweise im Shop benutzt 
werden können.\\
\newline
Im anfänglichen Spielbereich befindet sich außerdem eine Tür in der Mitte. 
Diese kann für 100 Münzen geöffnet werden und schaltet den Zugang zum Greenhouse frei. 
Dieses funktioniert wie das Haupthaus und bietet dem Spieler mehr Platz, um Pflanzen unterzubringen.\\
