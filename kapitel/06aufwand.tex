\chapter{Aufwand}\label{ch:aufwand}
Zu Beginn des Projekts wird eine Schätzung aller Aufwände geschätzt. 
Dies wird in der Gruppe von den Gruppenmitgliedern abgestimmt in der Einheit Stunden.
So kann einerseits die Arbeitseffizienz beurteilt werden und andererseits der Fortschritt des Projekts.
Jeder Aufwand wird von einem oder mehreren Gruppenmitgliedern geschätzt, die für die Aufgabe zuständig sind und 
die höchste Expertise im jeweiligen Bereich haben. 
Die Schätzung wird in Stunden angegeben und in einer Tabelle festgehalten.\\
\newline
Unter anderem wird der Aufwand die Konzeptionierung und Realisierung des Spiels geschätzt.
Darin kommen Aufwände auf, wie die Ideenfindung oder die Erstellung des Produktstrukturplans 
~\ref{fig:produktstrukturplan} und das Erstellen des Risikomanagements in Kapitel ~\ref{ch:risikomanagement}.
Zur Realisierung gehören unter anderem die Programmierung des Spiels und die Dokumentation auf Entwickler-
beziehungsweise Anwenderebene, welche unser Kanban auf Github und die Anleitung zur Verwendung des Spiels in Kapitel
~\ref{ch:anleitung}.\\
\newline
Besonders hoch werden die Aufwände für die Realisierung insgesamt und die Meetings innerhalb der Gruppe geschätzt.
Die Realisierung wird zu Anfang des Projekts auf 154 Stunden geschätzt, da die Programmierung des Spiels eine große Hürde
durch das Aneignen von Programmierkenntnissen in der verwendeten Programmiersprache und Engine darstellt.
Zusätzlich werden die Meetings auf insgesamt 48 Stunden geschätzt, weil einerseits gemeinsame Entwicklungssitzungen dazu
beitragen zeiteffizienter zu programmieren, jedoch auch zeitlichen Aufwand darstellen gewonnen Erkenntnisse in der
Gruppe zu teilen.
Dazu kommt es in einer Gruppe von fünf Personen zu häufigeren Meinungsverschiedenheiten, die Diskussionen und damit
zeitaufwändigere Meetings zur Folge haben.
Als weiterer größter Aufwand werden Teile des Spiels bewertet, die als Risiko eingestuft wurden und somit auf der einen 
Seite sehr viel Zeit in Anspruch nehmen könnten oder mit dem Ende des Projekts nicht vollständig bis gar nicht umgesetzt
werden. \\
\newline
In folgender Tabelle wird die Aufwandsschätzung dargestellt mit den Daten, um welchen Aufwand es sich handelt und der
Personenzuweisung, der Schätzung in Stunden, dem Realaufwand in Stunden und der Differenz zwischen Schätzung und
Realaufwand.
Das Vorzeichen der Differenz gibt Auskunft darüber, ob sich über- oder unterschätzt wurde.
Dabei ist ein positives Vorzeichen eine Überschätzung und ein negatives Vorzeichen eine Unterschätzung.\\
\newline
% Tabelle hier so:
% Tabelle hihihihihihi \\
% \newline
\newpage
In der Realität haben wir uns bei unseren Aufwänden in allen Bereichen überschätzt, also wir mussten weniger
Zeitaufwand investieren als ursprünglich geschätzt. 
Von der Sicht des Projektstarts ist das aber auch plausibel, denn die vorhandenen Herausforderungen aus Kapitel
~\ref{ch:herausforderungen}bezüglich der Programmierung und Aneignung von Kenntnissen zu Engine und Programmiersprache
sahen größer aus als sie in der Umsetzung waren.
Ebenfalls wurde der Workload von der Erstellung des Papers überschätzt, da zwar schon frühzeitig bekannt war es in Latex
zu erstellen, sich aber die Erarbeitung durch Arbeitsteilung der Paperautoren als weniger aufwändig herausstellte und 
die Menge der Texte einen kleineren Anteil des Papers darstellte als ursprünglich angenommen.
Implementierungen wie die Minispiele oder das Erstellen vieler Pflanzenassets für eine erhöhte Pflanzenvielfalt wurden 
als Risiken eingestuft und somit als aufwändiger angenommen, als sie in der Realität waren. 
Zur Risikobehandlung wurde die Minispielanzahl auf 1 bis 2 reduziert und nicht wie anfangs gedacht zum Beispiel auf 5 
festgelegt für mehr Spielvarietät, was den Aufwand und das Risiko erheblich reduzierte.
Die Pflanzenassets wurden durch den Designer erweitert, was durch die Einarbeitung des Designers im Pixelart-Editor 
nicht die erwartete Aufwandshöhe erreichte und damit genauso das Risiko minderte.\\ 