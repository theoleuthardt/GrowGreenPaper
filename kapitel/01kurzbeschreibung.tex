\chapter{Kurzbeschreibung}\label{sec:kurzbeschreibung}
\pagenumbering{arabic}
GrowGreen wird ein Singleplayer Gardening Game.\\
\textbf{Was soll gemacht werden?}\\
Es soll eine Spielumgebung geschaffen werden, die Nutzern ermöglicht ihre realen Pflanzen virtuell zu speichern und sich bestmöglich um sie zu kümmern. Pflanzen können für Coins aus einer begrenzten Auswahl gekauft werden. Coins können durch gute Pflege der Pflanzen, sowie mit Minispielen verdient werden. Gut gepflegte Pflanzen ermöglichen den Spielern, ihren Wohnraum zu vergrößern, oder sie für Coins zu verkaufen. Das reale Wetter soll sich im Spiel bemerkbar machen, dass z.B. bei Regen Pflanzen nicht gegossen werden müssen.\\
\textbf{Welche groben Ziele werden verfolgt?}\\
Das Spiel soll in Echtzeit stattfinden. Außerdem soll zu den gekauften Pflanzen reale Fakten abgebildet werden, wie Gießintervalle, Standortbedingungen sowie Eigenschaften. Während Spieler diese Intervalle abwarten, können sie beispielweise Minispiele spielen, um die Zeit zu verbringen und weitere Pflanzen kaufen zu können. Damit soll es nicht möglich sein zu viel Zeit im Spiel zu verbringen, um auch Zeit mit den realen Pflanzen verbringen zu können.\\
\textbf{Welchen Nutzen/Mehrwert bringt die Lösung für die Gesellschaft?}\\
Es bietet ein abwechslungsreiches Spiel für Pflanzenliebhaber, vor allem diese, die Schwierigkeiten bei der Wartung und Pflege ihrer Topfpflanzen haben bzw. davon träumen einen eigenen Garten in der Zukunft zu haben. Mithilfe von GrowGreen wird Nutzern erleichtert Verantwortung für Lebewesen zu übernehmen und Wissen zu ihren Pflanzen spielerisch aufbauen zu können.\\
\textbf{Wie ist die Vision?}\\
Wir wollen ein 2D Pixelart Spiel erstellen, welches Pflanzenliebhabern hilft, sich besser um ihre Pflanzen zu kümmern und in Welt der Pflanzen leichter einzutauchen. Wir werden eine leicht verständliche, intuitive Oberfläche schaffen. In ferner Zukunft kann eine Zusammenarbeit mit Blumenläden und Baumärkten möglich sein, für die Möglichkeit das Repertoire an Pflanzen, Blumen und Gartenzubehör zu erweitern. Zudem soll ein Verständnis für die Pflanzenwelt und Ökosysteme vermittelt werden. \\
\textbf{Wann ist das Projekt ein Erfolg?}\\
Das Projekt wird als Erfolg angesehen, sobald eine spielbare Version existiert. Dabei müssen Spielerdaten gespeichert und später wieder aufgerufen werden können. Bestenfalls wird das Projekt veröffentlicht und für Begeisterung in der Bevölkerung sorgen. \\