\chapter{Kurzbeschreibung}\label{ch:kurzbeschreibung}
\pagenumbering{arabic}
Pflanzen im Eigenheimoder Garten sorgen für ein natürliches Ambiente und ein geerdetes Umgebungsgefühl.\\
Doch nicht jeder hat die benötigten Gewohnheiten, um sich fachgerecht um seine Pflanzen zu kümmern.
GrowGreen soll dabei helfen, die Pflege von Pflanzen zu erleichtern und das Aneignen von Gewohnheiten 
zu einem spielerischen Erlebnis zu machen.\\[7pt]
% Was soll gemacht werden?
Ziel dieses Projekts ist es, eine Spielumgebung zu erschaffen, die Nutzern ermöglicht ihre realen Pflanzen auch virtuell zu besitzen 
und sich parallel in beiden Dimensionen bestmöglich um sie zu kümmern. 
Für Coins können eine begrenzte Auswahl der virtuellen Pflanzen gekauft werden, innerhalb des Shops. 
Pflegt der Spieler seine Pflanzen gut oder spielt in einem abgetrennten Modus Minispiele, verdient er damit Coins. 
Gut gepflegte Pflanzen ermöglichen den Spielern, ihren Wohnraum zu vergrößern. Die Spielfläche kann um ein
Gewächshaus erweitert werden.\\[7pt]
% Welche groben Zielen sollen verfolgt werden?
Zur Entwicklung von Gewohnheiten findet das Spiel in Echtzeit statt, so dass gleichzeitig auch die realen Pflanzen
außerhalb des Spiels wachsen und gedeihen.
Die virtuellen Pflanzen wachsen auch in Echtzeit und müssen regelmäßig gegossen und gepflegt werden.
Die Gießintervalle und Wachstumsphasen sind authentisch festgelegt.
Während Spieler diese Intervalle abwarten, können sie beispielweise Minispiele spielen, um die Zeit zu verbringen und 
Münzen zu verdienen. \\[7pt]
% Welchen Nutzen/Mehrwert bringt die Lösung für die Gesellschaft?
GrowGreen ist ein abwechslungsreiches Spiel für Pflanzenliebhaber, vor allem diese, die Schwierigkeiten bei der Wartung und 
Pflege ihrer Topfpflanzen haben. Die direkte Zielgruppe sind damit neurodiverse, vielbeschäftigte oder auch einfach vergessliche Personen. Wer von einen eigenen Garten träumt, ist hier genau richtig. 
Das Projekt erleichtert Nutzern Verantwortung für Lebewesen zu übernehmen und sich Wissen zu ihren Pflanzen 
spielerisch anzueignen. So werden nachhaltig Gewohnheiten zur der Pflanzenpflege entwickelt.\\[7pt]
% Wie ist die Vision?
Unser 2D Pixelart Spiel soll ein Verständnis für die Bedürfnisse und Ansprüche von gängigen Topfflanzen erwecken.
Zusätzlich zur Gewohnheitsbildung soll das Spiel einen Spaßfaktor bieten, um die langfristige Nutzung zu garantieren.
In ferner Zukunft ist eine Zusammenarbeit mit Blumenläden und Baumärkten in Ausblick, um
das verfügbare Repertoire an Pflanzen, Blumen und Gartenzubehör zu erweitern. \\[7pt]
% Wann ist das Projekt ein Erfolg?
Wir sehen GrowGreen als vollen Erfolg an, da eine spielbare Version mit allen grundlegend benötigten Hauptkomponenten 
erstellt und veröffentlicht wurde.
Dabei werden Spielerdaten gespeichert und können später wieder aufgerufen werden.
Bestenfalls wird das Projekt gut aufgenommen und für Begeisterung in der Bevölkerung sorgen.