\chapter{Kurzbeschreibung}\label{ch:kurzbeschreibung}
\pagenumbering{arabic}
Die meisten Menschen wollen in ihrem Eigenheim oder Garten Pflanzen halten für mehr Ambiente und ein natürlicheres 
Umgebungsgefühl. 
Doch nicht jeder hat die benötigten Gewohnheiten, um sich fachgerecht um die Pflanzen zu kümmern.
GrowGreen soll dabei helfen, die Pflege von Pflanzen zu erleichtern und das Aneignen von Gewohnheiten 
zu einem spielerischen Erlebnis zu machen.\\[7pt]
% Was soll gemacht werden?
Ziel dieses Projekts ist es, eine Spielumgebung zu erschaffen, die Nutzern ermöglicht ihre realen Pflanzen virtuell zu speichern 
und sich bestmöglich um sie zu kümmern. 
Die virtuellen Pflanzen können für Coins aus einer begrenzten Auswahl gekauft werden innerhalb des seperaten Shops. 
Pflegt der Spieler seine Pflanzen gut oder spielt in einem abgetrennten Modus Minispiele, verdient er damit Coins. 
Gut gepflegte Pflanzen ermöglichen den Spielern, ihren Wohnraum zu vergrößern mit einer Gebietserweiterung in Form eines
Gewächshauses.\\[7pt]
% Welche groben Zielen sollen verfolgt werden?
Zur besseren Entwicklung von Gewohnheiten findet das Spiel in Echtzeit statt, sodass gleichzeitig auch echte Pflanzen
außerhalb des Spiels wachsen und gedeihen.
Somit wachsen die virtuellen Pflanzen auch in Echtzeit und müssen regelmäßig gegossen und gepflegt werden.
Außerdem soll zu den gekauften Pflanzen reale Fakten abgebildet werden, wie Gießintervalle, Standortbedingungen 
sowie Eigenschaften für den Lernwert des Spiels. 
Während Spieler diese Intervalle abwarten, können sie beispielweise Minispiele spielen, um die Zeit zu verbringen und 
weitere Pflanzen kaufen zu können. 
Damit soll es nicht möglich sein zu viel Zeit im Spiel zu verbringen, um auch Zeit mit den realen Pflanzen 
verbringen zu können.\\[7pt]
% Welchen Nutzen/Mehrwert bringt die Lösung für die Gesellschaft?
Es bietet ein abwechslungsreiches Spiel für Pflanzenliebhaber, vor allem diese, die Schwierigkeiten bei der Wartung und 
Pflege ihrer Topfpflanzen haben beziehungsweise davon träumen einen eigenen Garten in der Zukunft zu haben. 
Mithilfe von GrowGreen wird Nutzern erleichtert Verantwortung für Lebewesen zu übernehmen und Wissen zu ihren Pflanzen 
spielerisch aufbauen zu können sowie gesunde Gewohnheiten bezüglich der Pflanzenpflege zu entwickeln.\\[7pt]
% Wie ist die Vision?
Wir wollen ein 2D Pixelart Spiel erstellen, welches Pflanzenliebhabern hilft, sich besser um ihre Pflanzen zu kümmern 
und in Welt der Pflanzen leichter einzutauchen. 
Damit soll ein Verständnis für die Pflanzenwelt und Ökosysteme vermittelt werden.
Zusätzlich zur Gewohnheitsbildung soll das Spiel einen Spaßfaktor bieten, um einen größeren Lerneffekt zu erzeugen.
In ferner Zukunft kann eine Zusammenarbeit mit Blumenläden und Baumärkten in Ausblick sein, für die Möglichkeit 
das Repertoire an Pflanzen, Blumen und Gartenzubehör zu erweitern. \\[7pt]
% Wann ist das Projekt ein Erfolg?
Das Projekt wird als Erfolg angesehen, sobald eine spielbare Version mit allen grundlegend benötigten Hauptkomponenten 
des Spiels existiert.
Dabei müssen Spielerdaten gespeichert und später wieder aufgerufen werden können.
Bestenfalls wird das Projekt veröffentlicht und für Begeisterung in der Bevölkerung sorgen.