\chapter{Qualitätskriterien}\label{ch:quality}
In diesem Abschnitt werden die Qualitätskriterien dieses Projektes näher betrachtet, um die Qualität des Spiels zu
bewerten und zu verbessern. 
Mithilfe der Auflistung klarer Anforderungen in einem Qualitätsregister kann zu Beginn des Projekts genau beschrieben
werden, welche Kriterien das Spiel bei der Fertigstellung erfüllen sollte, wie sie geprüft werden und 
ob diese abgeschlossen wurden zum Ende des Projekts.\\
\newline
Jede der Anforderungen besitzt eine Qualitätsnummer, Prüfmethode, Beschreibung, Toleranz mit den jeweiligen 
Verantwortlichen, die Termine zu planmäßiger und realem Abschlussdatum und dem Status.
Zur Gliederung aller Anforderungen hat jede eine Qualitätsnummer, die in aufsteigender Reihenfolge vergeben wurde.
Anhand der zugehörigen Prüfmethode wird festgelegt, wie die Anforderung nach erfolgreicher Realisierung geprüft wird, 
ob durch eine visuelle, praktische oder andere Prüfmethode.
Des Weiteren gibt die Beschreibung den Inhalt der Anforderung kurz und prägnant wieder, um die Anforderung verstehen 
zu können und damit der Durchführende die Prüfung der Anforderung umsetzen kann. 
Die Toleranz dient dem Zweck, wie viel Spielraum die Anforderung hat bei der Prüfung, um es als erfüllt gelten 
zu lassen durch das aktuell prüfende Gruppenmitglied.
Die Verantwortlichen sind die Personen, die für die Erfüllung der Anforderung verantwortlich gemacht werden könnten,
im Falle einer gescheiterten Realisierung oder Prüfung und dienen demnach dem Controlling.
Dagegen spiegelt die durchführenden Personen nur diejenigen wieder, die die Anforderung realisiert und geprüft haben.
Als Letztes werden noch Daten zu Planung und tatsächlichem Abschluss der Realisierung aller Anforderungen festgehalten,
um die anfängliche Vorstellung gegen die Realität der Umsetzung gegenüberzustellen.
Der Status hängt von der Erfüllung der Prüfmethode mit Betrachtung der Toleranz ab und bestimmt den Status der 
Anforderung, ob sie abgeschlossen wurde oder noch in Bearbeitung ist.\\
\newline
In Anhang %hier ref bitte
ist das Qualitätsregister zu finden, welches alle Qualitätskriterien des Spiels
enthält und die dazugehörigen Informationen zu den Anforderungen.