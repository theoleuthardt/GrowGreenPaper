\documentclass[12pt, a4paper, listof=totoc, bibliography=totoc, numbers=noenddot, ngerman, headsepline, oneside]{scrbook}
\usepackage{amsmath}
\usepackage[T1]{fontenc}
\usepackage{float}
\usepackage[utf8]{inputenc}
\usepackage[ngerman]{babel}
\usepackage{url}
\usepackage{graphicx} 
\usepackage{pdfpages}
\usepackage{longtable}
\usepackage{multirow}
\usepackage[a4paper, margin=1in]{geometry}
\usepackage[right]{eurosym} %Euro-Zeichen
\usepackage{amssymb}
\usepackage{subfig}
\usepackage{cite} %Quellenangaben
\usepackage{setspace} % Zeilenabstand
\usepackage[ 
   colorlinks,       
   linkcolor=black,   % Farbe interner Verweise 
   filecolor=black,   % Farbe externer Verweise 
   citecolor=black,   % Farbe von Zitaten 
   urlcolor=blue	  % Farbe von Links
   ]{hyperref} %Verlinkungen
\usepackage[figure]{hypcap}
\usepackage[ngerman]{translator}
\usepackage{blindtext} % Lorem-Ipsum-Plugin
\usepackage{glossaries}

\usepackage{listings,xcolor} %Codeanzeige
\usepackage{scrhack}
\usepackage[normalem]{ulem}
\useunder{\uline}{\ul}{}
\usepackage{wrapfig}

\usepackage{makecell}
\usepackage{comment}
\renewcommand{\familydefault}{\sfdefault} %%SCHRIFTART
\usepackage{pifont}

\graphicspath{ {./bilder/} }


\usepackage{chngcntr}
\usepackage{color}
\counterwithout{figure}{chapter}
\counterwithout{table}{chapter}

\definecolor{dkgreen}{rgb}{0,.6,0}
\definecolor{dkblue}{rgb}{0.337, 0.612, 0.839}
\definecolor{dkyellow}{rgb}{204, 255, 0}
\definecolor{lila}{rgb}{0.847, 0.624, 0.855}
\definecolor{hintergrund}{rgb}{0.118, 0.118, 0.118}
\definecolor{blau1}{rgb}{0.573, 0.820, 0.925}
\definecolor{orange}{rgb}{0.839, 0.616, 0.522}
\definecolor{psrot}{rgb}{0.66, 0.18, 0.00}
\definecolor{psblau}{rgb}{0, 0, 0.545}
\definecolor{psbg}{rgb}{0.9176, 0.9490, 0.9804}
\definecolor{psbg2}{rgb}{0.949, 0.949, 0.949}
\definecolor{psbg3}{rgb}{1.0, 0.9843, 0.9608}



\lstset{
    numbers=left, 
    numberstyle=\tiny\color{black}, 
    numbersep=5pt,
    breaklines=true,
    frame=lr,
    escapeinside={(*@}{@*)}, %nicht anzuzeigende Ausdrücke, z.B. für Labels
    language=[Sharp]C,
    showstringspaces=false,
    captionpos=b,
    backgroundcolor=\color{hintergrund},
    basicstyle=\ttfamily\fontsize{9}{10}\selectfont\color{white},
    keywordstyle    = \color{dkblue},
    stringstyle     = \color{orange},
    identifierstyle = \color{blau1},
    commentstyle    = \color{dkyellow},
    emph            =[1]{var},
    emphstyle       =[1]\color{dkblue},
    emph            =[2]{if,and,or,else, return},
    emphstyle       =[2]\color{lila}
    ƒ} 
\lstdefinestyle{cli}{
    basicstyle=\ttfamily\fontsize{9}{10}\selectfont\color{white},
    identifierstyle = \color{white},
    numbers=none,
    language=sh,
    stringstyle=\color{dkblue},
    emph            =[1]{dotnet},
    emphstyle       =[1]\color{dkyellow},
}
\lstdefinestyle{tiny}{
    basicstyle=\ttfamily\fontsize{7}{8}\selectfont\color{white},
    numbers=left, 
    numberstyle=\tiny\color{black}, 
    numbersep=5pt,
    breaklines=true,
    frame=lr,
    escapeinside={(*@}{@*)}, %nicht anzuzeigende Ausdrücke, z.B. für Labels
    language=[Sharp]C,
    showstringspaces=false,
    captionpos=t,
    backgroundcolor=\color{hintergrund},
    keywordstyle    = \color{dkblue},
    stringstyle     = \color{orange},
    identifierstyle = \color{blau1},
    commentstyle    = \color{dkyellow},
    emph            =[1]{var},
    emphstyle       =[1]\color{dkblue},
    emph            =[2]{if,and,or,else, return},
    emphstyle       =[2]\color{lila}
}
\lstdefinestyle{ps}{
    numbers=left, 
    numberstyle=\tiny\color{black}, 
    numbersep=5pt,
    breaklines=true,
    frame=none,
    escapeinside={(*@}{@*)}, %nicht anzuzeigende Ausdrücke, z.B. für Labels
    language=sh,
    showstringspaces=false,
    captionpos=b,
    backgroundcolor=\color{psbg3},
    basicstyle=\ttfamily\fontsize{9}{10}\selectfont\color{black},
    keywordstyle    = \color{psblau},
    stringstyle     = \color{psrot},
    identifierstyle = \color{psrot},
    commentstyle    = \color{dkyellow},
    emph            =[1]{var},
    emphstyle       =[1]\color{dkblue},
    emph            =[2]{if,and,or,else, return},
    emphstyle       =[2]\color{lila}
}
    

\renewcommand\lstlistingname{Codeausschnitt}
\renewcommand\lstlistlistingname{Codeverzeichnis}

\geometry{verbose,tmargin=2cm,bmargin=2cm,lmargin=2cm,rmargin=2.3cm} 

\clubpenalty = 10000 \widowpenalty = 10000 \displaywidowpenalty = 10000 

\newcommand{\footfigref}[1]{\footnote{Abb. \ref{#1} auf Seite \pageref{#1}}}

%% Bei Referenzen im Text wird jetzt bei allen Ebenen "Kapitel" vorgestellt, z.b. Kapitel 2, Kapitel 2.2, Kapitel 6.3.2
\addto\extrasngerman{%
    \def\sectionautorefname{Kapitel}%
    \def\subsectionautorefname{Kapitel}%
    \def\subsubsectionautorefname{Kapitel}%
    }

% Vertikaler Abstand zwischen Ende Textblock - Ende Fußzeile --> Abstand der Seitenzahl von Rand erhöhen 
\setlength{\footskip}{10mm}

\RedeclareSectionCommand[%
    beforeskip=0.5\baselineskip,
    afterskip=0.5\baselineskip
]{chapter}

\RedeclareSectionCommand[%
    beforeskip=0.5\baselineskip,
    afterskip=0.5\baselineskip
]{section}

\RedeclareSectionCommand[%
    beforeskip=0.1\baselineskip,
    afterskip=0.1\baselineskip
]{subsection}

\RedeclareSectionCommand[%
    beforeskip=0.1\baselineskip,
    afterskip=0.1\baselineskip
]{subsubsection}

\RedeclareSectionCommand[%
    beforeskip=0.01\baselineskip,
    %%afterskip=0.2\baselineskip
]{paragraph}

\setlength{\abovecaptionskip}{4pt}  % 1pc=12pt 
\setlength{\belowcaptionskip}{0pt}
%\setlength{\textfloatsep}{4pt}
\setlength{\intextsep}{1pc}

%% Verkleinerung der Textgröße unter Abbildungen
\addtokomafont{caption}{\small}

%%falsche Default-Silbentrennung überschreiben
\hyphenation{Soft-ware-ent-wick-lung}


\renewcommand*{\glspostdescription}{}

%Glossar-Befehle anschalten
%\makeglossaries
 
\KOMAoptions{parskip=full*}

% ändert Titelschriftart in Serifen-Normalschriftart
\addtokomafont{disposition}{\rmfamily} 

\makenoidxglossaries

\loadglsentries{glossar.tex}

\newcommand{\type}{Software Engineering I - Paper}
\newcommand{\topic}{Grow Green}
\newcommand{\subtopic}{Ein Pixelart-Pflanzenpflege-Spiel}
\newcommand{\Alex}{Alexander Betke (77203378972)}
\newcommand{\Maja}{Maja Günther (72205653284)}
\newcommand{\Theo}{Theo Leuthardt (77205844868)}
\newcommand{\Josh}{Josh Nicolai Tischer (77207542779)}
\newcommand{\Domi}{Domenik Wilhelm (77207300494)}
\newcommand{\gruppe}{2}
\newcommand{\jahrgang}{2023}
\newcommand{\fachbereich}{Duales Studium Wirtschaft · Technik}
\newcommand{\studiengang}{Informatik}
\newcommand{\modul}{IT2101 - Software Engineering I}
\newcommand{\betreuerHS}{Benjamin Kretzmer}
\newcommand{\wordCount}{2800}


\begin{document}
\author{}

\title{
\normalfont\endgraf\rule{\textwidth}{1pt}
\begingroup
	\centering
	\linespread{1.5}
	\huge\topic
\endgroup
\linespread{1.5}
\ \\ % Falls kein Subtopic, auskommentieren
\large\subtopic % Falls kein Subtopic, auskommentieren
\normalfont\endgraf\rule{\textwidth}{1pt}
}

\date{\large vorgelegt am 05. November 2024}

\publishers{
	\begin{tabular}{l l}
    \textbf{\normalsize{}} & \normalsize{}  \tabularnewline
    \textbf{\normalsize{}} & \normalsize{}  \tabularnewline
	\textbf{\normalsize{Modul:}} & \normalsize{\modul}  \tabularnewline
    \textbf{\normalsize{Gruppe:}} & \normalsize{\gruppe}  \tabularnewline
	\textbf{\normalsize{Namen:}} & \normalsize{\Alex}  \tabularnewline
 	\textbf{\normalsize{}} & \normalsize{\Maja}  \tabularnewline
   	\textbf{\normalsize{}} & \normalsize{\Theo}  \tabularnewline
    \textbf{\normalsize{}} & \normalsize{\Josh}  \tabularnewline
    \textbf{\normalsize{}} & \normalsize{\Domi}  \tabularnewline
	\textbf{\normalsize{Studienjahrgang:}} & \normalsize{\jahrgang}  \tabularnewline
	\textbf{\normalsize{Fachbereich:}} & \normalsize{\fachbereich} \tabularnewline
	\textbf{\normalsize{Studiengang:}} & \normalsize{\studiengang} \tabularnewline
	\textbf{\normalsize{Betreuerin Hochschule:}} & \normalsize{\betreuerHS} \tabularnewline
    \textbf{\normalsize{Anzahl der Wörter:}} & \normalsize{\wordCount} \tabularnewline
    \tabularnewline
    \tabularnewline
	\end{tabular}
    \begin{tabular}{p{4.6em} p{0.3em} p{4.6em} p{0.3em} p{4.6em} p{0.3em} p{4.6em} p{0.3em} p{4.6em}}
        \hspace{4.5cm} && \hspace{4.5cm} && \hspace{4.5cm} && \hspace{4.5cm} && \hspace{4.5cm}\\
        \hspace{4.5cm} && \hspace{4.5cm} && \hspace{4.5cm} && \hspace{4.5cm} && \hspace{4.5cm}\\
        \cline{1-1}\cline{3-3}\cline{5-5}\cline{7-7}\cline{9-9} 
        \small{Alexander Betke} && \small{Maja Günther} && \small{Theo Leuthardt} && \small{Josh Tischer} && \small{Domenik Wilhelm}
    \end{tabular}
	}

\titlehead{\begin{center}
    \includegraphics[height=0.031\textheight]{bilder/HWR.png}
    \hfill
    \includegraphics[height=0.031\textheight]{bilder/Polizei.png}
    \hfill
    \includegraphics[height=0.023\textheight]{bilder/FUBIT.png}
    \hfill
    \includegraphics[height=0.026\textheight]{bilder/BDR.jpg}
    \hfill
    \includegraphics[height=0.029\textheight]{bilder/Alstom.png}
    \end{center}
    }

\maketitle
\onehalfspacing 
\pagenumbering{Roman}

\clearpage

\newpage

\tableofcontents{}
\addcontentsline{toc}{chapter}{Inhaltsverzeichnis}

\clearpage

\addcontentsline{toc}{chapter}{Akronyme}
\printnoidxglossary[type=\acronymtype]
\clearpage

\addcontentsline{toc}{chapter}{Glossar}
\printnoidxglossary
\clearpage

\chapter{Kurzbeschreibung}\label{ch:kurzbeschreibung}
\pagenumbering{arabic}
Die meisten Menschen wollen in ihrem Eigenheim oder Garten Pflanzen halten für mehr Ambiente und ein natürlicheres 
Umgebungsgefühl. 
Doch nicht jeder hat die benötigten Gewohnheiten, um sich fachgerecht um die Pflanzen zu kümmern.
GrowGreen soll dabei helfen, die Pflege von Pflanzen zu erleichtern und das Aneignen von Gewohnheiten 
zu einem spielerischen Erlebnis zu machen.\\[7pt]
% Was soll gemacht werden?
Ziel dieses Projekts ist es, eine Spielumgebung zu erschaffen, die Nutzern ermöglicht ihre realen Pflanzen virtuell zu speichern 
und sich bestmöglich um sie zu kümmern. 
Die virtuellen Pflanzen können für Coins aus einer begrenzten Auswahl gekauft werden innerhalb des seperaten Shops. 
Pflegt der Spieler seine Pflanzen gut oder spielt in einem abgetrennten Modus Minispiele, verdient er damit Coins. 
Gut gepflegte Pflanzen ermöglichen den Spielern, ihren Wohnraum zu vergrößern mit einer Gebietserweiterung in Form eines
Gewächshauses.\\[7pt]
% Welche groben Zielen sollen verfolgt werden?
Zur besseren Entwicklung von Gewohnheiten findet das Spiel in Echtzeit statt, sodass gleichzeitig auch echte Pflanzen
außerhalb des Spiels wachsen und gedeihen.
Somit wachsen die virtuellen Pflanzen auch in Echtzeit und müssen regelmäßig gegossen und gepflegt werden.
Außerdem soll zu den gekauften Pflanzen reale Fakten abgebildet werden, wie Gießintervalle, Standortbedingungen 
sowie Eigenschaften für den Lernwert des Spiels. 
Während Spieler diese Intervalle abwarten, können sie beispielweise Minispiele spielen, um die Zeit zu verbringen und 
weitere Pflanzen kaufen zu können. 
Damit soll es nicht möglich sein zu viel Zeit im Spiel zu verbringen, um auch Zeit mit den realen Pflanzen 
verbringen zu können.\\[7pt]
% Welchen Nutzen/Mehrwert bringt die Lösung für die Gesellschaft?
Es bietet ein abwechslungsreiches Spiel für Pflanzenliebhaber, vor allem diese, die Schwierigkeiten bei der Wartung und 
Pflege ihrer Topfpflanzen haben beziehungsweise davon träumen einen eigenen Garten in der Zukunft zu haben. 
Mithilfe von GrowGreen wird Nutzern erleichtert Verantwortung für Lebewesen zu übernehmen und Wissen zu ihren Pflanzen 
spielerisch aufbauen zu können sowie gesunde Gewohnheiten bezüglich der Pflanzenpflege zu entwickeln.\\[7pt]
% Wie ist die Vision?
Wir wollen ein 2D Pixelart Spiel erstellen, welches Pflanzenliebhabern hilft, sich besser um ihre Pflanzen zu kümmern 
und in Welt der Pflanzen leichter einzutauchen. 
Damit soll ein Verständnis für die Pflanzenwelt und Ökosysteme vermittelt werden.
Zusätzlich zur Gewohnheitsbildung soll das Spiel einen Spaßfaktor bieten, um einen größeren Lerneffekt zu erzeugen.
In ferner Zukunft kann eine Zusammenarbeit mit Blumenläden und Baumärkten in Ausblick sein, für die Möglichkeit 
das Repertoire an Pflanzen, Blumen und Gartenzubehör zu erweitern. \\[7pt]
% Wann ist das Projekt ein Erfolg?
Das Projekt wird als Erfolg angesehen, sobald eine spielbare Version mit allen grundlegend benötigten Hauptkomponenten 
des Spiels existiert.
Dabei müssen Spielerdaten gespeichert und später wieder aufgerufen werden können.
Bestenfalls wird das Projekt veröffentlicht und für Begeisterung in der Bevölkerung sorgen.
\chapter{Projektsteckbrief}\label{ch:steckbrief}
% bitti Maja
\chapter{Ziele}\label{ch:ziele}
Um den Erfolg des Projekts messbar zu machen, werden in diesem Kapitel die Projektziele von Grow Green definiert.
Nach mehreren Bearbeitungen und Veränderungen ergaben sich folgende Ziele.
\begin{table}[]
    \centering
    \begin{tabular}{ccccc}
        \hline
        \multicolumn{1}{|c|}{\textbf{Nr.}} & \multicolumn{1}{c|}{\textbf{Klassifizierung}} & \multicolumn{1}{c|}{\textbf{Beschreibung}}                                                                                                                        & \multicolumn{1}{c|}{\textbf{Messkriterium}}                                                                                                                                                                  & \multicolumn{1}{c|}{\textbf{Priorität}} \\ \hline
        1                                  & Leistung                                      & Hauptszene                                                                                                                                                        & \begin{tabular}[c]{@{}c@{}}Hintergrund und Logik zu Spielerbewegung sowie\\ Pflanzenplazierlogik wurde implementiert.\end{tabular}                                                                           & Muss                                    \\
        2                                  & Leistung                                      & Titelmenü (Startmenü des Spiels)                                                                                                                                  & \begin{tabular}[c]{@{}c@{}}Zugehörige Menüpunkte zum Starten, Laden eines \\ Spiels, zur Charakterauswahl und zu den Einstellungen \\ wurden erstellt, dessen Logik implementiert und getestet.\end{tabular} & Muss                                    \\
        3                                  & Leistung                                      & Shopszene                                                                                                                                                         & \begin{tabular}[c]{@{}c@{}}Logik zum Kaufen und Verkaufen von Pflanzen, Kauf von\\ Pflanzensamen und Shopinterface wurde implementiert \\ und getestet.\end{tabular}                                         & Muss                                    \\
        4                                  & Leistung                                      & Weltszene                                                                                                                                                         & \begin{tabular}[c]{@{}c@{}}Szene sammt Szenenwechsellogik und -animation wurde \\ implementiert und getestet.\end{tabular}                                                                                   & Kann                                    \\
        5                                  & Leistung                                      & \begin{tabular}[c]{@{}c@{}}Implementierung der Pflanzenlogik bezüglich\\ Bewässerung und Pflanzenwachstum\end{tabular}                                            & \begin{tabular}[c]{@{}c@{}}Realistische Funktionalität wurde nach der Implementierung\\ getestet.\end{tabular}                                                                                               & Muss                                    \\
        6                                  & Leistung                                      & \begin{tabular}[c]{@{}c@{}}Backend in Form einer Datenbank \\ zur Speicherung aller Spieler- und Pflanzendaten\end{tabular}                                       & \begin{tabular}[c]{@{}c@{}}Spieler und Pflanzendaten werden korrekt gespeichert und \\ ausgelesen, Debuggingtools dienen zum Testen der Werte.\end{tabular}                                                  & Muss                                    \\
        7                                  & Leistung                                      & \begin{tabular}[c]{@{}c@{}}Bereitstellung von Github Actions zum automatisierten \\ Kompilieren des Papers sowie des Spiels \\ als ausführbare Datei\end{tabular} & \begin{tabular}[c]{@{}c@{}}Testweises Kompilieren der aktuellen Stände von Paper\\ und Spielprojekt auf Github\end{tabular}                                                                                  & Muss                                    \\
        8                                  & Leistung                                      & \begin{tabular}[c]{@{}c@{}}Texturen für Pflanzen, Töpfe und\\ weitere Assets des Spiels\end{tabular}                                                              & \begin{tabular}[c]{@{}c@{}}Speicherung aller benötigten Assets auf Github mit \\ entsprechender Verfügbarkeit in der Spielengine\end{tabular}                                                                & Muss                                    \\
        9                                  & Leistung                                      & Logoerstellung                                                                                                                                                    & \begin{tabular}[c]{@{}c@{}}Nutzung eines fertigen Logos in allen Facetten des Spiel\\ mit der Zufriedenheit aller Gruppenmitglieder\end{tabular}                                                             & Soll                                    \\
        10                                 & Hauptziel                                     & Fertigstellung des Papers                                                                                                                                         & \begin{tabular}[c]{@{}c@{}}Alle Kapitel wurden fertig gestellt mit zusätzlicher \\ Unterzeichnung der ehrenwörtlichen Erklärung aller\\ Gruppenmitglieder.\end{tabular}                                      & Muss                                    \\
        11                                 & Termin                                        & Fertigstellung der Spieleentwicklung                                                                                                                              & \begin{tabular}[c]{@{}c@{}}Das Spiel wird am 29.10.2024 um 23:59 Uhr fertig\\ gestellt in seiner Entwicklung.\end{tabular}                                                                                   & Kann                                    \\
        12                                 & Termin                                        & Endpräsentation                                                                                                                                                   & \begin{tabular}[c]{@{}c@{}}Vorstellung des Projekts bei der Endpräsentation mit \\ zusätzlicher Abgabe der Folien bei Moodle \\ bis zum 29.10.2024 um 23:59 Uhr\end{tabular}                                 & Muss                                    \\
        13                                 & Termin                                        & Abschluss des Projektes                                                                                                                                           & \begin{tabular}[c]{@{}c@{}}Pünktliche Abgabe des Source-Codes und der \\ Endpäsentation auf Moodle am 29.10.2024 \\ um 23:59 Uhr und des Papers am 05.11.2024 \\ um 23:59 Uhr\end{tabular}                   & Soll                                    \\
        14                                 & Leistung                                      & Sehr gute Bewertung des Projekts                                                                                                                                  & Gewünschte Bewertung vom Stakeholder wird erreicht                                                                                                                                                           & Soll                                    \\
        15                                 & Leistung                                      & \begin{tabular}[c]{@{}c@{}}Lerneffekt bei Nutzern des Spiels zum Wissen über\\ Pflanzen und Ökosysteme\end{tabular}                                               & Feedback von Nutzern des Spiels                                                                                                                                                                              & Soll                                    \\
        16                                 & Sozial                                        & \begin{tabular}[c]{@{}c@{}}Entwicklung von Gewohnheiten bei der Pflanzenpflege\\ der Haus- oder Gartenpflanzen des Nutzers\end{tabular}                           & Feedback von Nutzern des Spiels                                                                                                                                                                              & Soll                                    \\
        17                                 & Leistung                                      & Nutzer haben Spaß am Spielen von Grow Green                                                                                                                       & Feedback von Nutzern des Spiels                                                                                                                                                                              & Kann
    \end{tabular}\label{tab:projektziele}
\end{table}
\chapter{Aufgaben - Kompetenzen - Verantwortlichkeiten (AKV)}\label{sec:AKV}
\chapter{Risikomanagement}\label{ch:risikomanagement}
Um Unwissenheiten im Entwicklungsprozess zu minimieren, ist es ein essenzieller Schritt mögliche Risiken frühzeitig 
zu erkennen, die den Projektablauf entweder positiv als Chance oder negativ als Bedrohung beeinflussen können.
Durch die Identifikation und Bewertung dieser Risiken können Maßnahmen ergriffen werden, um die negativen
Einflüsse zu minimieren und die positiven Einflüsse zu maximieren. \\
\newline
Im Anhang A ist ein Risikoregister zu finden, das für die Einordnung möglicher Risiken erstellt wird. 
In diesem Register wird eine erarbeitete Sammlung an Risiken aufgeführt mit einer zugehörigen Nummer sowie einer
Beschreibung zur näheren Erläuterung.
Jedes Risiko besitzt eine eigene Eintrittswahrscheinlichkeit in Prozent mit einer Auswirkungsanalyse in den Graden 
gering, mittel oder groß, die die Bedrohlichkeit beziehungsweise Möglichkeit des Risikos darstellt.
Die Kategorisierung der Risiken unterteilt sich in die zwei Aspekte Art und Typ des Risikos, wobei der Typ den Einfluss
in positiver oder negativer Form, also Chance oder Bedrohung darstellt.
Abschließend wird jedem Risiko eine Behandlung mit zugehöriger Beschreibung beigefügt, damit Unwissenheiten bei der 
Entwicklung beseitigt werden und der Abschluss des Projekts sichergestellt werden kann. \\
\newline
Zur optimalen Veranschaulichung werden alle Risiken des Registers in einer Risikomatrix dargestellt, die die
Eintrittswahrscheinlichkeit gegen die Auswirkung aufträgt.
So kann auf einen Blick ein Risiko mit zum Beispiel starker Bedrohlichkeit erkannt werden und die entsprechende 
Behandlung eingeleitet werden. \\
\chapter{Aufwand}\label{ch:aufwand}
Zu Beginn des Projekts wird eine Schätzung aller Aufwände geschätzt. 
Dies wird in der Gruppe von den Gruppenmitgliedern abgestimmt in der Einheit Stunden.
So kann einerseits die Arbeitseffizienz beurteilt werden und andererseits der Fortschritt des Projekts.
Jeder Aufwand wird von einem oder mehreren Gruppenmitgliedern geschätzt, die für die Aufgabe zuständig sind und 
die höchste Expertise im jeweiligen Bereich haben. 
Die Schätzung wird in Stunden angegeben und in einer Tabelle festgehalten.\\
\newline
Unter anderem wird der Aufwand die Konzeptionierung und Realisierung des Spiels geschätzt.
Darin kommen Aufwände auf, wie die Ideenfindung oder die Erstellung des Produktstrukturplans 
~\ref{fig:produktstrukturplan} und das Erstellen des Risikomanagements in Kapitel ~\ref{ch:risikomanagement}.
Zur Realisierung gehören unter anderem die Programmierung des Spiels und die Dokumentation auf Entwickler-
beziehungsweise Anwenderebene, welche unser Kanban auf Github und die Anleitung zur Verwendung des Spiels in Kapitel
~\ref{ch:anleitung}.\\
\newline
Besonders hoch werden die Aufwände für die Realisierung insgesamt und die Meetings innerhalb der Gruppe geschätzt.
Die Realisierung wird zu Anfang des Projekts auf 154 Stunden geschätzt, da die Programmierung des Spiels eine große Hürde
durch das Aneignen von Programmierkenntnissen in der verwendeten Programmiersprache und Engine darstellt.
Zusätzlich werden die Meetings auf insgesamt 48 Stunden geschätzt, weil einerseits gemeinsame Entwicklungssitzungen dazu
beitragen zeiteffizienter zu programmieren, jedoch auch zeitlichen Aufwand darstellen gewonnen Erkenntnisse in der
Gruppe zu teilen.
Dazu kommt es in einer Gruppe von fünf Personen zu häufigeren Meinungsverschiedenheiten, die Diskussionen und damit
zeitaufwändigere Meetings zur Folge haben.
Als weiterer größter Aufwand werden Teile des Spiels bewertet, die als Risiko eingestuft wurden und somit auf der einen 
Seite sehr viel Zeit in Anspruch nehmen könnten oder mit dem Ende des Projekts nicht vollständig bis gar nicht umgesetzt
werden. \\
\newline
In folgender Tabelle wird die Aufwandsschätzung dargestellt mit den Daten, um welchen Aufwand es sich handelt und der
Personenzuweisung, der Schätzung in Stunden, dem Realaufwand in Stunden und der Differenz zwischen Schätzung und
Realaufwand.
Das Vorzeichen der Differenz gibt Auskunft darüber, ob sich über- oder unterschätzt wurde.
Dabei ist ein positives Vorzeichen eine Überschätzung und ein negatives Vorzeichen eine Unterschätzung.\\
\newline
% Tabelle hier so:
% Tabelle hihihihihihi \\
% \newline
In der Realität haben wir uns bei unseren Aufwänden in allen Bereichen überschätzt, also wir mussten weniger
Zeitaufwand investieren als ursprünglich geschätzt. 
Von der Sicht des Projektstarts ist das aber auch plausibel, denn die vorhandenen Herausforderungen aus Kapitel
~\ref{ch:herausforderungen}bezüglich der Programmierung und Aneignung von Kenntnissen zu Engine und Programmiersprache
sahen größer aus als sie in der Umsetzung waren.
Ebenfalls wurde der Workload von der Erstellung des Papers überschätzt, da zwar schon frühzeitig bekannt war es in Latex
zu erstellen, sich aber die Erarbeitung durch Arbeitsteilung der Paperautoren als weniger aufwändig herausstellte und 
die Menge der Texte einen kleineren Anteil des Papers darstellte als ursprünglich angenommen.
Implementierungen wie die Minispiele oder das Erstellen vieler Pflanzenassets für eine erhöhte Pflanzenvielfalt wurden 
als Risiken eingestuft und somit als aufwändiger angenommen, als sie in der Realität waren. 
Zur Risikobehandlung wurde die Minispielanzahl auf 1 bis 2 reduziert und nicht wie anfangs gedacht zum Beispiel auf 5 
festgelegt für mehr Spielvarietät, was den Aufwand und das Risiko erheblich reduzierte.
Die Pflanzenassets wurden durch den Designer erweitert, was durch die Einarbeitung des Designers im Pixelart-Editor 
nicht die erwartete Aufwandshöhe erreichte und damit genauso das Risiko minderte.\\ 
\chapter{Produktstruktur}\label{sec:produktstruktur}
Vollständiger Produkstrukturplan! WICHTIG!
\chapter{Herausforderungen und Lösungen}\label{ch:herausforderungen}
% Projektthema und Technologieauswahl
Die erste Herausforderungen zu Beginn des Projekts bestand für das Team darin, das sich auf ein Thema und eine Stilart
geeinigt wurde, da verschiedene Erfahrungs- und Wissensstände der Teammitglieder zu Projektideen mit unterschiedlichen 
Schwierigkeitsstufen geführt hätten. 
So wurde beschlossen, ein 2D-Pixelart-Spiel zu entwickeln, anstatt von der 3D-Spieleentwicklung Gebrauch zu machen, weil 
nur Josh damit Erfahrung besaß und die Entwicklung eines solchen Spiels mit zugehörigen Texturen und Modellen
eine zu hohen zeitlichen Aufwand dargestellt hätte. 
Entscheidend war zudem die umfangreiche Recherche zur Auswahl der Engine zur Entwicklung des Spiels. 
Kriterien dafür waren eine hoffentlich steile Lernkurve mit dem Engine-Editor, die Möglichkeit zur Programmierung mit 
weitläufig verbreiteten Programmiersprache und eine gute Dokumentation zur effizienteren Einarbeitung und Beseitigung
von Fehlern. 
Mit der Nutzung der Open-Source-Engine GODOT und der Programmiersprache C\# wurden alle der geforderten 
Kriterien erfüllt, die für die Entwicklung des Projekts erforderlich sind. \\
\newline
% Frontend
Beim Frontend des Spiels lief der Entwicklungsprozess relativ reibungslos ab unter anderem durch den nutzerfreundlichen 
GODOT-Editor.
Viel mehr kamen Schwierigkeiten auf durch das Verwenden und Manipulieren von Objekten aus Godot in C\#.
In jeder Szene des Spiels existiert eine Hierarchie der sich darin befindenden Objekte, die die Struktur der Szene
darstellen. 
Bei dem Versuch zum Beispiel das Spiel zu schließen, wenn im Titelmenü der Ausschaltbutton gedrückt wird, muss im 
zugewiesenen Skript das Spiel geschlossen werden. 
Nur stellte sich die Frage, wie wir in C\# auf die Godot-Objekte zugreifen können. 
Durch die integrierte Godot-System-Klasse kann auf die zuvor genannte Hierarchie zugegriffen werden über relative Pfade
und so als verwendbares Objekt manipuliert werden kann oder wie im genannten Beispiel das Spiel durch einen 
systemspezifischen Befehl geschlossen werden kann.\\
\newline
% Backend
Damit während des Spielens die Daten des Nutzers und alle Pflanzendaten nicht beim Schließen des Spiels 
verloren gehen, entstand das Problem der Datenspeicherung mithilfe eines Backends. 
Möglich wäre die Nutzung von verschiedenste Backed-Frameworks gewesen wie ASP .NET oder gängige 
Datenbank-Wrapper-Systeme mit Javascript oder Java wie NodeJS oder Spring-Boot.
Da wir aber kein großes Spiel mit Skalierbarkeit entwickeln werden, ist es am effizientesten eine Datenbank in SQL zu 
definieren und das Wrapper-System in C\# direkt zu implementieren. 
Zusätzlicher Vorteil dabei war die Erfahrung von Alex und Maja bezüglich Datenbanken, was zu einer zügigen Entwicklung
des Backends führte.\\
\newline
\newpage
% Charaterauswahl
In Verbindung mit dem Backend erwies es sich als etwas schwieriger, die Charakterauswahl aus dem Titelmenü in anderen
Spielszenen zu speichern und demnach die Identifizierung der Auswahl.
Als Lösung wurde jedem Charakter eine ID zugeordnet, die in der Datenbank gespeichert wird und bei benötigtem Aufruf
in der jeweiligen Klasse durch den Datenbank-Wrapper geladen wird.
Die Assets der Charaktere enthalten ihre ID im Dateinamen, wodurch das Laden über GODOT deutlich vereinfacht ist. \\
\newline
% Organisation
Im generellen Sinne kamen die meisten Probleme in der Organisation der Entwicklung und Zusammenarbeit auf. 
Unter anderem regelmäßig Termine zu finden für Besprechungen außerhalb der Vorlesungen und einen einheitlichen
Kommunikationskanal zu haben, damit Vorstellungen zur Entwicklung des Spiels jedes Gruppenmitglieds global in der Gruppe
bekannt sind.
Zudem fehlte es anfangs, dass jeder über seine Aufgaben und Pflichten Bescheid weiß, was zu ineffizienter Arbeitsweise 
führte.
Um einen geordneten Entwicklungsprozess zu erhalten, entschieden wir uns Aufgaben und die Zuweisung von Bearbeitern 
in der Gruppe über ein Github-Kanban-System zu lösen.
Darin kann jedes Gruppenmitglied Issues erstellen, also sprich Aufgaben, die Features des Spiels oder ähnliche 
projektbezogene To-dos darstellen und Gruppenmitgliedern zugewiesen werden können.
Über die Kommentarfunktion innerhalb der Issues kann auch ein Austausch zu Problemen stattfinden,
wodurch die Kommunikation nicht nur auf die Besprechungen begrenzt ist.
Stellt ein Gruppenmitglied seine Aufgabe fertig, trägt jener seine benötigte Zeit in einer Excel-Tabelle ein, damit
die Arbeitszeit am Projekt verfolgt werden kann.
\chapter{UML}\label{sec:uml}
\chapter{Qualitätskriterien}\label{ch:quality}
\chapter{Technologien und Produkte}\label{ch:technologien}
Zur Entwicklung des Projekts Grow Green werden Technologien verwendet, die der erleichterten Arbeitsweise dienen
und die Produktivität der Gruppe steigern sollen. 
Im Folgenden werden die verwendeten Technologien näher erläutert. \\
\newline
% Godot & C#
Zur Entwicklung des Spiels ist die initiale Auswahl der Engine sehr bedeutend. 
Wir haben uns für die Engine Godot in der 4.\ Version entschieden, da es sich um ein Open-Source-Projekt der Godot 
Foundation handelt und damit die Anforderung der Kostenfreiheit erfüllt.
Durch die Unterstützung der Programmiersprache C\#, neben der Godot-spezifischen Skriptsprache, stand Godot im Vergleich
zu anderen Engines nicht schlechter da und biete die Möglichkeit bisher erworbenes Wissen aus dem letzten Semester in 
objektorientierter Programmierung anwenden und erweitern zu können.
Zum leichteren Debugging sollte unsere verwendete Engine eine gute Dokumentation haben, was Godot für seine aktuelle 
Version in großen Umfang pflegt und ebenfalls ein entscheidendes Kriterium darstellte.
Zudem bietet Godot eine hohe Kompatibilität an bezüglich des späteren Exports des Spiels, womit das Spiel im 
Deployment für alle großen Betriebssysteme exportiert werden kann. \\
\newline
% SQL
Für das Backend kommt SQL zum Einsatz in Form einer Datenbank.
Dies stellt für uns die einfachste Möglichkeit dar, Daten verständlich zu speichern, ohne ein großes Backendframework
lernen zu müssen und damit einen größeren Zeitaufwand zu haben. 
Ebenfalls kann wie bei C\# aus den vergangenen Semestern gelerntes Wissen angewendet werden und somit zusätzliche 
Mehrarbeit vermieden werden.
% IDE
% Git & Github
% Github Actions
% Pixilart (Assets pixeln)
% Latex
\chapter{Anleitung zur Verwendung}\label{sec:anleitung}

\clearpage
\pagenumbering{Alph}

\bibliography{literatur}

\bibliographystyle{hwrbib}


\listoffigures
\lstlistoflistings

\listoftables
\addchap{Anhang}


\chapter*{Ehrenwörtliche Erklärungen}
\addcontentsline{toc}{chapter}{Ehrenwörtliche Erklärung}


Wir erklären ehrenwörtlich:
\begin{enumerate}
	\item dass wir unser Paper selbstständig verfasst habe,
	\item dass wir die Übernahme wörtlicher Zitate aus der Literatur sowie die Verwendung der Gedanken anderer Autoren an den entsprechenden Stellen innerhalb der Arbeit gekennzeichnet habe,
	\item dass wir unser Paper bei keiner anderen Prüfung vorgelegt haben.
\end{enumerate}
Wir sind uns bewusst, dass eine falsche Erklärung rechtliche Folgen haben wird.\\
\\ \\
\begin{tabular}{lp{2em}l} 
 \hspace{4cm}   && \hspace{7cm} \\\cline{1-1}\cline{3-3} 
 \small{Ort, Datum} && \small{Alexander Betke} \\
 \\\\
 \hspace{4cm}   && \hspace{7cm} \\\cline{1-1}\cline{3-3} 
 \small{Ort, Datum} && \small{Maja Günther} \\
 \\\\
  \hspace{4cm}   && \hspace{7cm} \\\cline{1-1}\cline{3-3} 
 \small{Ort, Datum} && \small{Theo Leuthardt} \\
 \\\\
  \hspace{4cm}   && \hspace{7cm} \\\cline{1-1}\cline{3-3} 
 \small{Ort, Datum} && \small{Josh Nicolai Tischer} \\
 \\\\
  \hspace{4cm}   && \hspace{7cm} \\\cline{1-1}\cline{3-3} 
 \small{Ort, Datum} && \small{Domenik Wilhelm} \\
\end{tabular}

\end{document}
